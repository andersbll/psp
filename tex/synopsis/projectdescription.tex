\documentclass[10pt,a4paper,final,oneside,openany,article]{memoir}
% [12pt, 11pt, landscape, titlepage, a4paper, draft, twoside, oneside, 
%  openleft, openright, openany]

\chapterstyle{hangnum}

%BASIC PACKAGES
\usepackage{eso-pic,fix-cm,ae,aecompl,ifthen}         
\usepackage[danish,english]{babel} % last language decides document language!
\usepackage[utf8x]{inputenc}          %text encoding
\usepackage{amsmath,amssymb, amsbsy}  % math
\usepackage{graphicx}
\usepackage[usenames,dvipsnames]{color}
\usepackage[british]{isodate}
%\usepackage{morefloats}

\newsubfloat{figure}

%MISC. PACKAGES
%\usepackage{multicol}        % \begin{multicols}{2} \end{multicols}
%\usepackage{array}           % advanced tables
%\usepackage{multirow}        % advanced tables
%\usepackage{longtable}       % split tables over pages
%\usepackage{textcomp}        % symbols
%\usepackage{verbatim}        % monospace code environment
%\usepackage{pdflscape}       % \begin{landscape} \end{landscape}
%\usepackage{semantic}        % Good for proof trees and math ligatures
%\usepackage[noend]{algorithmic}
\usepackage{microtype}        % AWESOME typography!
\usepackage{colortbl}
\usepackage{marvosym}
%Kan bruges som \fixme{blabla}. Vises sålænge tex doc er i draft.
\usepackage[draft] { fixme } 


%FONT
\usepackage[T1]{fontenc}
\usepackage{palatino}              % font : garamond
\linespread{1.05}                  % Palatino needs more leading (space between lines)
\usepackage{bera}
%\renewcommand{\ttdefault}{cmtt}    % alternative monospace font
%\renewcommand{\rmdefault}{ugm}
\usepackage[sc]{mathpazo}
%\usepackage{euler}                % weirdo math
%PAGE DIMENSIONS

%\usepackage[left=4.5cm, right=4.5cm, top=4.4cm, bottom=4.5cm]{geometry}


%HEADERS
\makepagestyle{myheadings}

%\makepsmarks{myheadings}{
%  \def\chaptermark##1{\markboth{%
%    \ifnum \value{secnumdepth} < -1
%      \if@mainmatter
%        \chaptername\ \thechapter\ --- %
%      \fi
%    \fi
%    ##1}{}}

%  \def\sectionmark##1{\markright{%
%    \ifnum \value{secnumdepth} < 0
%      \thesection. \ %
%    \fi
%    ##1}}
%}
%%\makeevenhead{myheadings}{\thechapter\hskip.3cm\vrule\hskip.3cm \leftmark}{}{}
%\makeoddhead{myheadings}{}{}{\leftmark\hskip.3cm\vrule\hskip.3cm\thechapter}
%%\makeoddhead{myheadings}{}{}{\rightmark\hskip.3cm\vrule\hskip.3cm\thesection}
%\makeevenfoot{myheadings}{}{\thepage}{}
%\makeoddfoot{myheadings}{}{\thepage}{}
%\pagestyle{myheadings}

% customize chapter pages

\makepagestyle{myheadingschapterpage}
  \makeevenfoot{myheadingschapterpage}{}{}{\thepage}
  \makeoddfoot{myheadingschapterpage}{}{}{\thepage}
\aliaspagestyle{chapter}{myheadingschapterpage}
\aliaspagestyle{title}{myheadingschapterpage}
\makeevenfoot{myheadings}{}{}{\thepage}
\makeoddfoot{myheadings}{}{}{\thepage}
\pagestyle{myheadings}



%PDF OUTPUT
\usepackage{hyperref}             % clickable url's in PDF-output
\hypersetup{
%    unicode=true,          % non-Latin characters in Acrobat’s bookmarks
%    pdftitle={My title},    % title
%    pdfauthor={Author},     % author
%    pdfsubject={Subject},   % subject of the document
%    pdfcreator={Creator},   % creator of the document
%    pdfproducer={Producer}, % producer of the document
%    pdfkeywords={keywords}, % list of keywords
%    pdfnewwindow=true,      % links in new window
    colorlinks=true,       % false: boxed links; true: colored links
    linkcolor=black,          % color of internal links
    citecolor=black,        % color of links to bibliography
    filecolor=black,      % color of file links
    urlcolor=black           % color of external links
}




%PRETTY COLORS
\usepackage{color}
\definecolor{blue}{rgb}{0,0,0.8}
\definecolor{green}{rgb}{0,0.5,0}
\definecolor{red}{rgb}{0.5,0,0}
\definecolor{grey}{rgb}{0.5,0.5,0.5}


%SECTION TITLES
\def\thefigure{\arabic{figure}}
%\def\thesection{\arabic{section}}
%\def\thesubsection{\thesection.\arabic{subsection}}
%\def\thesubsubsection{\alph{subsubsection}.}
% \alph \roman \arabin
%\setcounter{secnumdepth}{3}
%\setcounter{tocdepth}{3}


% USER DEFINED COMMANDS AND ENVIRONMENTS
%\newcommand{\codevar}[1]{{\tt{\it #1}}}
%\newcommand{\genericleft}{\langle\hspace{-2.6pt}\vert}  % prints [[
%\newcommand{\genericright}{\vert\hspace{-2.6pt}\rangle} % prints ]]
%\newcommand{\generic}[1]{\genericleft #1 \genericright^{\varepsilon}}

%FIGURE CAPTIONS
%\usepackage[labelformat=empty]{subfig}
\usepackage{sidecap} % side captions: \begin{SCfigure}[2.7][ht] ...
\usepackage{caption}
\captionsetup{margin=0pt, font=small, labelfont=bf, format=hang}
\setlength{\abovecaptionskip}{0pt}
\setlength{\belowcaptionskip}{0pt}

%LINE SPACING
%\usepackage{setspace}
%EXAMPLE:
% \singlespacing, \onehalfspacing, \doublespacing, \setstretch{x}


%PROGRAM CODE WITH HIGLIGHTING AND SHAZZ!
%\usepackage{listings}
\usepackage{amsmath}


%DOCUMENT INFO
\title{
  Side-chain packing \\
  \small{Project description}
}
\author{
	Martin Dybdal -- \texttt{dybber@dybber.dk}\\
	Anders Boesen Lindbo Larsen -- \texttt{abll@diku.dk} \\
    Esben Skaarup -- \texttt{sben@sben.dk}
}
%\datebritish
\date{\today}

\newcommand{\subimgwidth}{.48\textwidth}
\newcommand{\imgwidth}{.85\textwidth}
\renewcommand\vec[1]{\boldsymbol{#1}}
\setcounter{secnumdepth}{0}
			
\begin{document}
\maketitle

\section{Problem statement}
In this project we want to develop a model for representing protein structures including all atoms of the molecule.
Such a model is called an all-atom model.
Using this model, we will develop a method for predicting side-chain
conformations of a protein model given its three-dimensional backbone
structure.
With this method we will investigate whether the problem gets simplified in terms of the number computation steps (e.g. number of collision detetections between atom pairs) when the given three-dimensional backbone structure is closer to the physically correct structure of the protein.
Furthermore, we want to experiment with the prediction method to see if any features of the protein structure can be utilize to simplify/improve the prediction.


\section{Motivation}
The algorithms team at our department have implemented a protein
structure prediction algorithm based solely on the $C_\alpha$ atoms of
the protein backbone. In an attempt to get their model closer to the
actual structure of physical proteins, they need a way to add the
missing parts to the protein-backbone\footnote{Ultimately, we want to
  do the missing steps to be able to compete in the
  CASP competition.}.

Experiments performed by the algorithms team has shown that there
could possibly be a connection between how \textit{easy} it is to add
the side-chains to the model and how good their solution to the
backbone-problem is. Thus, adding side-chains and other missing atoms
can possibly contribute to the computation of the tertiary structure
of the backbone, by rejecting certain conformations where adding
side-chains is hard.

\section{Elaboration}

\begin{figure}
  \centering
  \subbottom[The $C_{\alpha}$-backbone]{
    \includegraphics{Calpha_backbone}  
    \label{fig:Calpha_backbone}
  }
  \subbottom[The connection of amino acids in the protein backbone]{
    \includegraphics{amino_connect}  
    \label{fig:amino_connect}
  }
\end{figure}


The building stone of proteins are amino acids\footnote{A detailed
  introduction to the structure of proteins is found in
  \cite{branden}}. A protein consist of at least one linear chain of
amino acids. In Figure \ref{fig:amino_connect} we have shown how a
sequence of amino acids are connected. A carbon atom is located at the
center of each amino acid, this is called the $C_\alpha$ atom. The
largest variability of the protein structure is found in the angles
between the individual $C_\alpha$ atoms. Thus, the approach taken by
the algorithms team at our department is to describe the protein
structure solely by the $C_\alpha$ atoms, in the attempt to do protein
structure prediction.

When a structure of the $C_\alpha$ atoms is found, we will add the missing atoms to construct a model of the complete protein, including side chains. Apart from obtaining the position of all of the atoms, we can use this to measure how well the backbone and side chains fit together, and thereby get a more detailed indicator of how accurate the backbone prediction is.


Adding side chains to a folded backbone may introduce new problems. An
added side chain can contribute with additional free energy, making the
conformation unstable, and at worst causing collisions
with other parts of the backbone or other side chains.


The stability of a protein is determined by the amount of free energy in the structure. 
The goal of the protein structure prediction is thus to find the most plausible protein structure by minimizing the free energy.
A realistic energy calculation will require an insight in biochemistry that is beyond the scope of this project. 
Therefore, we limit ourselves to measure the sum of root mean square deviations, RMSD, on the Euclidean distance between each atom of the true protein structure and our structure prediction.



\section{Learning goals}
Our learning goals for this project includes:
\begin{itemize}
\item Models for representing protein structures and side-chains
\item Basics of protein structure prediction
% \item Apply linear algebra to represent rotations of protein structures
\item Apply transformations on protein model using linear algebra
\end{itemize}


\section{Limitations}
\begin{itemize}
\item Interconnections between side chain atoms from two different
  aminogroups, such as a disulfid-bridge between two side chains.
\item Energy computations
\end{itemize}

\bibliographystyle{apalike}
\bibliography{../bibliography}


% Spørgsmål til Rasmus:
%  - Energiberegning, Metropolis, Boltzmann etc.
%  - Leonard Jones potentialet

\end{document}

