The evaluation of our solution has revealed a couple of weaknesses that should be addressed. 

From the Ramachandran plot in Figure \ref{fig:eval_ramachandran_fitted} we see that the angles generated by the CCD method are not always realistic.
This could be improved by introducing Ramachandran probability maps.
A probability map is used to to accept or reject sets of $\phi$ and $\psi$ angles proposed by the CCD-algorithm.
If a new set $\phi,\psi$ is proposed, the angles are chosen if they have higher probability than the current $\phi,\psi$.
If not, the angles are chosen with probability $p_{\text{new}}/p_{\text{old}}$. 
Probability maps for all amino acid types are available online and should be straight forward to integrate \cite{10.1371/journal.pcbi.1000763}.
Moreover, the use of probability maps has the benefit that the problem with collisions between the proline side-chain and the backbone is most likely to be eliminated as the illegal combination of angles is improbable. 

Another problem occurring when determining side-chain conformations is the inevitable collisions caused by an unfavorable backbone folding. 
To accommodate this situation, it could be relevant to investigate the possibility of using information from the side-chain structures to alter the backbone accordingly. 

So far, we have neglected the $\omega$ angle as $\omega$ angles different from 180$^\circ$ rarely occur. 
However, when this happen our backbone fitting method is penalized as the $\omega$ angle is locked.
If we should take $\omega$ into consideration, it would be inconvenient to include it in the CCD method like $\phi$ and $\psi$ as it is only supposed to be adjusted on rare occasions.
Instead, an $\omega$ test could be carried out on the entire backbone when it has been fitted to a low RMSD.
For a backbone with low RMSD, an incorrect $\omega$ angle will stand out.

Finally, we have mainly experimented with real proteins. 
It would be interesting to do work with the synthesized \Ca-traces as these might introduce new difficulties.


%\section{Experiment with Hydrogen atoms}
%How large is the variability of the hydrogen atoms in the backbone and
%side-chains? How hard would it be to add hydrogen atoms on to an
%otherwise full model of the protein? If the hydrogen atoms are easy to
%add later, we could post-pone the problem of adding them and thereby
%simplifying our model.


%%% Local Variables: 
%%% mode: latex
%%% TeX-master: "rapport"
%%% End: 
