\documentclass[10pt,a4paper,final,oneside,openany,article, twocolumn]{memoir}
% [12pt, 11pt, landscape, titlepage, a4paper, draft, twoside, oneside, 
%  openleft, openright, openany]

\chapterstyle{article}

%BASIC PACKAGES
\usepackage{eso-pic,fix-cm,ae,aecompl,ifthen}         
\usepackage[danish,english]{babel} % last language decides document language!
\usepackage[utf8x]{inputenc}          %text encoding
\usepackage{amsmath,amssymb, amsbsy}  % math
\usepackage{graphicx}
\usepackage[usenames,dvipsnames]{color}
\usepackage[british]{isodate}
%\usepackage{morefloats}
\usepackage[style=alphabetic,natbib=true]{biblatex}
\usepackage{amsmath}
\usepackage[sc]{mathpazo}
\usepackage{MnSymbol}

%\bibliographystyle{apalike}
\bibliography{../bibliography}


\newsubfloat{figure}

%MISC. PACKAGES
%\usepackage{multicol}        % \begin{multicols}{2} \end{multicols}
%\usepackage{array}           % advanced tables
%\usepackage{multirow}        % advanced tables
%\usepackage{longtable}       % split tables over pages
%\usepackage{textcomp}        % symbols
%\usepackage{verbatim}        % monospace code environment
%\usepackage{pdflscape}       % \begin{landscape} \end{landscape}
%\usepackage{semantic}        % Good for proof trees and math ligatures
%\usepackage[noend]{algorithmic}
\usepackage{microtype}        % AWESOME typography!
%\usepackage{colortbl}
%\usepackage{marvosym}
%Kan bruges som \fixme{blabla}. Vises sålænge tex doc er i draft.
\usepackage[draft]{fixme}


%FONT
\usepackage[T1]{fontenc}
\usepackage{palatino}              % font : garamond
\linespread{1.05}                  % Palatino needs more leading (space between lines)
\usepackage{bera}
%\renewcommand{\ttdefault}{cmtt}    % alternative monospace font
%\renewcommand{\rmdefault}{ugm}
%\usepackage{euler}                % weirdo math
%PAGE DIMENSIONS

%\usepackage[left=4.5cm, right=4.5cm, top=4.4cm, bottom=4.5cm]{geometry}


%HEADERS
\makepagestyle{myheadings}

%\makepsmarks{myheadings}{
%  \def\chaptermark##1{\markboth{%
%    \ifnum \value{secnumdepth} < -1
%      \if@mainmatter
%        \chaptername\ \thechapter\ --- %
%      \fi
%    \fi
%    ##1}{}}

%  \def\sectionmark##1{\markright{%
%    \ifnum \value{secnumdepth} < 0
%      \thesection. \ %
%    \fi
%    ##1}}
%}
%%\makeevenhead{myheadings}{\thechapter\hskip.3cm\vrule\hskip.3cm \leftmark}{}{}
%\makeoddhead{myheadings}{}{}{\leftmark\hskip.3cm\vrule\hskip.3cm\thechapter}
%%\makeoddhead{myheadings}{}{}{\rightmark\hskip.3cm\vrule\hskip.3cm\thesection}
%\makeevenfoot{myheadings}{}{\thepage}{}
%\makeoddfoot{myheadings}{}{\thepage}{}
%\pagestyle{myheadings}

% customize chapter pages

\makepagestyle{myheadingschapterpage}
  \makeevenfoot{myheadingschapterpage}{}{}{\thepage}
  \makeoddfoot{myheadingschapterpage}{}{}{\thepage}
\aliaspagestyle{chapter}{myheadingschapterpage}
\aliaspagestyle{title}{myheadingschapterpage}
\makeevenfoot{myheadings}{}{}{\thepage}
\makeoddfoot{myheadings}{}{}{\thepage}
\pagestyle{myheadings}



%PDF OUTPUT
\usepackage{hyperref}             % clickable url's in PDF-output
\hypersetup{
%    unicode=true,          % non-Latin characters in Acrobat’s bookmarks
%    pdftitle={My title},    % title
%    pdfauthor={Author},     % author
%    pdfsubject={Subject},   % subject of the document
%    pdfcreator={Creator},   % creator of the document
%    pdfproducer={Producer}, % producer of the document
%    pdfkeywords={keywords}, % list of keywords
%    pdfnewwindow=true,      % links in new window
    colorlinks=true,       % false: boxed links; true: colored links
    linkcolor=black,          % color of internal links
    citecolor=black,        % color of links to bibliography
    filecolor=black,      % color of file links
    urlcolor=black           % color of external links
}




%PRETTY COLORS
\usepackage{color}
\definecolor{blue}{rgb}{0,0,0.8}
\definecolor{green}{rgb}{0,0.5,0}
\definecolor{red}{rgb}{0.5,0,0}
\definecolor{grey}{rgb}{0.5,0.5,0.5}


%SECTION TITLES
\def\thefigure{\arabic{figure}}
%\def\thesection{\arabic{section}}
%\def\thesubsection{\thesection.\arabic{subsection}}
%\def\thesubsubsection{\alph{subsubsection}.}
% \alph \roman \arabin
%\setcounter{secnumdepth}{3}
\setcounter{tocdepth}{3}


% USER DEFINED COMMANDS AND ENVIRONMENTS
%\newcommand{\codevar}[1]{{\tt{\it #1}}}
%\newcommand{\genericleft}{\langle\hspace{-2.6pt}\vert}  % prints [[
%\newcommand{\genericright}{\vert\hspace{-2.6pt}\rangle} % prints ]]
%\newcommand{\generic}[1]{\genericleft #1 \genericright^{\varepsilon}}

%FIGURE CAPTIONS
%\usepackage[labelformat=empty]{subfig}
\usepackage{sidecap} % side captions: \begin{SCfigure}[2.7][ht] ...
\usepackage{caption}
\captionsetup{margin=0pt, font=small, labelfont=bf, format=hang}
\setlength{\abovecaptionskip}{0pt}
\setlength{\belowcaptionskip}{0pt}

%LINE SPACING
%\usepackage{setspace}
%EXAMPLE:
% \singlespacing, \onehalfspacing, \doublespacing, \setstretch{x}


%PROGRAM CODE WITH HIGLIGHTING AND SHAZZ!
%\usepackage{listings}



%DOCUMENT INFO
\title{\vspace{-2cm}
  Fitting an All-atom Protein Model to a $C_\alpha$-trace\\
}
\author{
	Martin Dybdal -- \texttt{dybber@dybber.dk}\\
	Anders Boesen Lindbo Larsen -- \texttt{abll@diku.dk} \\
    Esben Skaarup -- \texttt{sben@sben.dk}
}

%\datebritish
\date{\today}

\newcommand{\subimgwidth}{.48\textwidth}
\newcommand{\imgwidth}{.85\textwidth}
\renewcommand\vec[1]{\boldsymbol{#1}}
\newcommand{\rotateAround}[1]{\lcirclearrowright \hspace{-3mm}\colorbox{white}{}{} \hspace{-1.7mm} _{\scriptscriptstyle #1}}
\setcounter{secnumdepth}{0}
\setcounter{chapter}{0}
%\renewcommand\thesection{\arabic{section}}


			
\begin{document}
\twocolumn[
  \begin{@twocolumnfalse}
    \maketitle
\begin{abstract}
  When predicting the structure of proteins, it is common simplify the
  problem by using a model that just includes a few of the protein
  molecules. In particular, the algorithms group at our department
  only predicts a folding of the $C_\alpha$-trace, thus not including
  most backbone atoms and all side-chain molecules. In this report we
  will investigate a strategy for predicting the structure of a
  protein including all its atoms (an all-atom model), using an
  already folded $C_\alpha$-trace as target.
  \vspace{0.5cm}
\end{abstract}
  \end{@twocolumnfalse}
]

\tableofcontents

\chapter{Introduction}
Proteins is perhaps the most important building block of living
organisms. They a multitude of tasks and is found in all lifeforms,
from bacteria and unicellular organisms to multicellular organisms
such as animals. The structure of proteins determine their function. When looking at the large amount of functions
performed by proteins it should be clear, that understanding how their
structure is determined is an important research field.

% \fixme{noget med designer drugs, m.m. og hvorfor vi er interesserede
%   i at forudsige den tertiære struktur}

The algorithms group at our department have implemented a protein
structure prediction algorithm based solely on the $C_\alpha$ atoms of
the protein backbone\cite{}. In an attempt to get their model closer
to the actual structure of physical proteins, they needed a way to add
the missing parts to the protein, to get an all-atom model.  Also,
supplying an all-atom model is a requirement for participating in the
CASP experiment\cite{caspwebsite}.

\section{Protein structure}
The building stone of proteins are amino acids\footnote{A detailed
  introduction to the structure of proteins is found in
  \cite{branden}}. A protein consists of one or several chains
(polymers) of amino acids. In Figure \ref{fig:amino_connect} we have
shown how a sequence of amino acids is connected. The carbon atom
connecting the backbone with the side-chains is called the $C_\alpha$
atom. The largest variability of the protein structure is found in the
angles between the individual $C_\alpha$ atoms \cite{lotan04},
\textit{the torsion angles}. Thus, the approach taken by the
algorithms group at our department is to describe the protein
structure solely by the $C_\alpha$ atoms (see Figure
\ref{fig:Calpha_backbone}), in the attempt to do protein structure
prediction.	

The $C_\alpha$-trace generated by this method is used as target in our
fitting of the all-atom model. The fitting should be conducted, such
that it minimizes the number of clashes and at the same time minimizes
the deviation from the target $C_\alpha$-trace.

The stability of a protein is determined by the amount of free energy
in the structure. The goal of the protein structure prediction is thus
to find the most plausible protein structure by minimizing the free
energy. A realistic energy calculation will require an insight in
biochemistry that is beyond the scope of this project.  Therefore, we
limit ourselves to minimize the number of clashes as well as the
deviation from the provided $C_\alpha$-trace.

\section{Protein Structure Prediction}
Ab initio vs. denovo
Rotamers
RMSD


\section{Related Work}


\chapter{All-atom Backbone}

$||N - C|| \rotateAround{180^\circ} \rotateAround{90^\circ} \rotateAround{v}$


\section{Inverse kinematics}

\chapter{Handling Side-chains}



\chapter{Evaluation}

\chapter{Conclusion}

\defbibheading{bibliography}[\bibname]{%
  \chapter{#1}%
  \markboth{#1}{#1}}
\printbibliography

\end{document}

