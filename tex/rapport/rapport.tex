\documentclass[10pt,a4paper,final,oneside,openany,article,twocolumn]{memoir}
% [12pt, 11pt, landscape, titlepage, a4paper, draft, twoside, oneside, 
%  openleft, openright, openany]

%\chapterstyle{article}

%BASIC PACKAGES
\usepackage{eso-pic,fix-cm,ae,aecompl,ifthen}         
\usepackage[danish,english]{babel} % last language decides document language!
\usepackage[utf8x]{inputenc}          %text encoding
\usepackage{amsmath,amssymb, amsbsy}  % math
\usepackage{graphicx}
\usepackage[usenames,dvipsnames]{color}
\usepackage[british]{isodate}
%\usepackage{morefloats}
\usepackage[style=alphabetic,natbib=true]{biblatex}
\usepackage{amsmath}
\usepackage[sc]{mathpazo}
\usepackage{MnSymbol}

%\bibliographystyle{apalike}
\bibliography{../bibliography}


\newsubfloat{figure}

%MISC. PACKAGES
%\usepackage{multicol}        % \begin{multicols}{2} \end{multicols}
%\usepackage{array}           % advanced tables
%\usepackage{multirow}        % advanced tables
%\usepackage{longtable}       % split tables over pages
%\usepackage{textcomp}        % symbols
%\usepackage{verbatim}        % monospace code environment
%\usepackage{pdflscape}       % \begin{landscape} \end{landscape}
%\usepackage{semantic}        % Good for proof trees and math ligatures
%\usepackage[noend]{algorithmic}
\usepackage{microtype}        % AWESOME typography!
%\usepackage{colortbl}
%\usepackage{marvosym}
%Kan bruges som \fixme{blabla}. Vises sålænge tex doc er i draft.
\usepackage[draft]{fixme}


%FONT
\usepackage[T1]{fontenc}
\usepackage{palatino}              % font : garamond
\linespread{1.05}                  % Palatino needs more leading (space between lines)
\usepackage{beramono}
%\renewcommand{\ttdefault}{cmtt}    % alternative monospace font
%\renewcommand{\rmdefault}{ugm}
%\usepackage{euler}                % weirdo math
%PAGE DIMENSIONS

%\usepackage[left=4.5cm, right=4.5cm, top=4.4cm, bottom=4.5cm]{geometry}


%HEADERS
\makepagestyle{myheadings}

%\makepsmarks{myheadings}{
%  \def\chaptermark##1{\markboth{%
%    \ifnum \value{secnumdepth} < -1
%      \if@mainmatter
%        \chaptername\ \thechapter\ --- %
%      \fi
%    \fi
%    ##1}{}}

%  \def\sectionmark##1{\markright{%
%    \ifnum \value{secnumdepth} < 0
%      \thesection. \ %
%    \fi
%    ##1}}
%}
%%\makeevenhead{myheadings}{\thechapter\hskip.3cm\vrule\hskip.3cm \leftmark}{}{}
%\makeoddhead{myheadings}{}{}{\leftmark\hskip.3cm\vrule\hskip.3cm\thechapter}
%%\makeoddhead{myheadings}{}{}{\rightmark\hskip.3cm\vrule\hskip.3cm\thesection}
%\makeevenfoot{myheadings}{}{\thepage}{}
%\makeoddfoot{myheadings}{}{\thepage}{}
%\pagestyle{myheadings}

% customize chapter pages

\makepagestyle{myheadingschapterpage}
  \makeevenfoot{myheadingschapterpage}{}{}{\thepage}
  \makeoddfoot{myheadingschapterpage}{}{}{\thepage}
\aliaspagestyle{chapter}{myheadingschapterpage}
\aliaspagestyle{title}{myheadingschapterpage}
\makeevenfoot{myheadings}{}{}{\thepage}
\makeoddfoot{myheadings}{}{}{\thepage}
\pagestyle{myheadings}



%PDF OUTPUT
\usepackage{hyperref}             % clickable url's in PDF-output
\hypersetup{
%    unicode=true,          % non-Latin characters in Acrobat’s bookmarks
%    pdftitle={My title},    % title
%    pdfauthor={Author},     % author
%    pdfsubject={Subject},   % subject of the document
%    pdfcreator={Creator},   % creator of the document
%    pdfproducer={Producer}, % producer of the document
%    pdfkeywords={keywords}, % list of keywords
%    pdfnewwindow=true,      % links in new window
    colorlinks=true,       % false: boxed links; true: colored links
    linkcolor=black,          % color of internal links
    citecolor=black,        % color of links to bibliography
    filecolor=black,      % color of file links
    urlcolor=black           % color of external links
}




%PRETTY COLORS
\usepackage{color}
\definecolor{blue}{rgb}{0,0,0.8}
\definecolor{green}{rgb}{0,0.5,0}
\definecolor{red}{rgb}{0.5,0,0}
\definecolor{grey}{rgb}{0.5,0.5,0.5}


%SECTION TITLES
\def\thefigure{\arabic{figure}}
%\def\thesection{\arabic{section}}
%\def\thesubsection{\thesection.\arabic{subsection}}
%\def\thesubsubsection{\alph{subsubsection}.}
% \alph \roman \arabin
\setcounter{secnumdepth}{1}
\setcounter{tocdepth}{1}


% USER DEFINED COMMANDS AND ENVIRONMENTS
%\newcommand{\codevar}[1]{{\tt{\it #1}}}
%\newcommand{\genericleft}{\langle\hspace{-2.6pt}\vert}  % prints [[
%\newcommand{\genericright}{\vert\hspace{-2.6pt}\rangle} % prints ]]
%\newcommand{\generic}[1]{\genericleft #1 \genericright^{\varepsilon}}

%FIGURE CAPTIONS
%\usepackage[labelformat=empty]{subfig}
\usepackage{sidecap} % side captions: \begin{SCfigure}[2.7][ht] ...
\usepackage{caption}
\captionsetup{margin=0pt, font=small, labelfont=bf, format=hang}
\setlength{\abovecaptionskip}{0pt}
\setlength{\belowcaptionskip}{0pt}

%LINE SPACING
%\usepackage{setspace}
%EXAMPLE:
% \singlespacing, \onehalfspacing, \doublespacing, \setstretch{x}


%PROGRAM CODE WITH HIGLIGHTING AND SHAZZ!
%\usepackage{listings}



%DOCUMENT INFO
\title{\vspace{-2cm}
  Fitting an All-atom Protein Model to a $C_\alpha$-trace\\
}
\author{
	Martin Dybdal -- \texttt{dybber@dybber.dk}\\
	Anders Boesen Lindbo Larsen -- \texttt{abll@diku.dk} \\
	Esben Skaarup -- \texttt{sben@sben.dk}
}

%\datebritish
\date{\today}

\newcommand{\subimgwidth}{.48\textwidth}
\newcommand{\imgwidth}{.85\textwidth}
\renewcommand\vec[1]{\boldsymbol{#1}}
\newcommand{\Ca}{C$_\alpha$}
\newcommand{\rotateAround}[1]{\lcirclearrowright \hspace{-3mm}\colorbox{white}{}{} \hspace{-1.7mm} _{\scriptscriptstyle #1}}
\setcounter{secnumdepth}{0}
\setcounter{chapter}{0}
%\renewcommand\thesection{\arabic{section}}

\twocoltocetc
\setlength{\absleftindent}{1.5cm}
\setlength{\absrightindent}{\absleftindent}
			
\begin{document}
\twocolumn[\maketitle
\begin{onecolabstract}
  Protein structure prediction is often simplified by using a model
  that includes only few of the atoms actually present in proteins. In
  particular, the algorithms group at our department only predicts a
  folding of the $C_\alpha$-trace, and are thus excluding most
  backbone atoms and all side-chain molecules.

  In this report we will investigate a strategy for predicting the
  native structure of proteins including all atoms (an all-atom
  model). Our method uses an already folded $C_\alpha$-trace as target.
  \vspace{1.5cm}
\end{onecolabstract}
]
\tableofcontents*

\newpage

\ \\

\hspace{0.5cm}\includegraphics[width=0.4\textwidth]{figures/forside.png}


\newpage
\chapter{Introduction}
Proteins is perhaps the most important molecules of living
organisms. They perform a multitude of biological tasks and is found
in all lifeforms, from bacteria and unicellular organisms to
multicellular organisms such as animals. The chemical abilities and
biological functions of proteins is determined by their
three-dimensional structure. The ability to determine this structure
without performing costly experiments will have many applications in
medicine (e.g. drug design) and biotechnology. Protein structure
prediction and the related topic protein folding\footnote{The two
  research fields are distinguished by whether the actual folding
  process is simulated (protein folding) or a legal structure is
  sought without computing the intermediate steps (protein structure
  prediction).} is large an active research fields. It is still an
open problem.

\section{Protein structure}
The building blocks of proteins are amino acids\footnote{A detailed
  introduction to the structure of proteins is found in
  \cite{branden}}. There are twenty different amino acids which all
share a common structure. The type of amino acid is identified by an
attached molecule called the side-chain. A protein consists of one or
several unbranched chains (polymers) of amino acids. The order of
amino acids in such a chain is the primary structure of the
protein. For any given sequence of amino acids it is hypothesised
(Anfinsens dogma, \cite{anfinsen73, soundararajan2010}) that there is
a single unique and stable three-dimensional structure with minimum
amount of free energy. This is known as the native structure of the
protein and determining it computationally is the problem of protein
structure prediction.

In Figure \ref{fig:amino_connect} we have shown how a sequence of
amino acids are connected. The carbon atom connecting the backbone
with the side-chains is called the $C_\alpha$-atom.  The largest
variability of the protein structure is found in the angles between
the individual $C_\alpha$-atoms \cite{lotan04}, \textit{the torsion
  angles}. The approach to structure prediction taken by the
algorithms group at our department is thus to describe the protein
structure solely by the $C_\alpha$-atoms of the protein
backbone\fxnote{cite}. Such a sequence of $C_{\alpha}$ atoms is called a
$C_{\alpha}$-trace (see Figure \ref{fig:Calpha_backbone}).


\begin{figure}
  \centering
  \subbottom[$C_{\alpha}$-trace]{
    \includegraphics[width=0.48\textwidth]{figures/Calpha_backbone}  
    \label{fig:Calpha_backbone}
  }
  \subbottom[All-atom protein backbone, with $R_1$, $R_2$ and $R_3$ representing side-chains]{
    \includegraphics[width=0.48\textwidth]{figures/amino_connect}  
    \label{fig:amino_connect}
  }
  \caption{}
\end{figure}

\section{Protein structure prediction}
The chemical stability of a protein molecule is determined by the
amount of free energy in its structure. The goal of protein structure
prediction is to find the most plausible protein structure by
minimizing the free energy. 

% Why simplifications are necessary to make the problem
% computationally feasible.

One approach to protein structure prediction is to use statistical
information of proteins with known native structure. This is called as
\textit{comparative modelling}. Another approach, known as \textit{ab
  initio} or \textit{de novo}, starts ``from scratch'' in the sense,
that no information apart from the chemical interactions between atoms
are used. This is the approach taken by the algorithms group at our
department.

% Rotamers
% RMSD



% Limitation: A realistic energy calculation will require an insight in biochemistry
% that is beyond the scope of this project.  Therefore, we limit
% ourselves to minimize the number of clashes as well as the deviation
% from the provided $C_\alpha$-trace.



\section{Our strategy}
In an attempt to get the $C_\alpha$ model closer to the actual
structure of physical proteins, we will in this project try to add the
missing atoms to the protein, to get an all-atom model.  This will
also enable our algorithms group to to participate in the CASP
experiment \cite{caspwebsite}.

The $C_\alpha$-trace generated by their prediction algorithm is used
as target in our fitting of the all-atom model. The fitting should be
conducted, such that it minimizes the number of clashes and at the
same time minimizes the deviation from the target $C_\alpha$-trace.

Our fitting problem can be considered as two somewhat separate problems.
First, the backbone must be fitted to the \Ca-trace. 
Hereafter, the amino acid residues are added to the backbone changing the backbone conformation only if necessary.
In the following we have chosen to consider these two tasks separately even though the residue handling will cause changes in the backbone.

In Section \ref{sec:fitting_backbone}, we try to fit the all-atom backbone to the \Ca-trace ignoring the amino acid residues.
In Section \ref{sec:handling_sidechains}, we try to add the side-chains to the all-atom backbone changing the backbone-conformation only if necessary.

%By considering these tasks separately, we simplify our solution since ... \fixme{begrund hvorfor}
%Fitting the backbone while taking side-chain placement into consideration 
%%Her vil vi gå igennem vores overordnede strategi:


\section{Related Work}
Todo: snak om SCRWL, der tilføjer side-chains på en fastlåst backbone.

\chapter{Fitting the all-atom backbone}
\label{sec:fitting_backbone}
The first step in our fitting algorithm is to fit the protein backbone to the given \Ca-trace.


Her vil vi forklarer det første trin vi udfører: folder backbonen, med
alle atomer hen til $C_\alpha$-tracet.

$||N - C|| \rotateAround{180^\circ} \rotateAround{90^\circ} \rotateAround{v}$


\section{Inverse kinematics}
Lidt om invers-kinematik.

\chapter{Handling Side-chains}
\label{sec:handling_sidechains}
Hvordan vi tilføjer side-chains og derefter tilpasser hele kæden igen.

\chapter{Evaluation}
Evaluering af metoden med kørsel på diverse kendte proteiner (og
evt. ukendte?)

\chapter{Conclusion}

%\defbibheading{bibliography}[\bibname]{%
%  \chapter{#1}%
%  \markboth{#1}{#1}}
\printbibliography

\end{document}

